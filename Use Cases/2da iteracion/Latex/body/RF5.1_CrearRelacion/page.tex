\phantomsection\numberedsection{RF5.1 Crear relación}

\subsection*{Descripción}
Permite a los usuarios definir nuevas relaciones para poder relacionar productos con otros y sugerirlos de una forma más rápida.\par
\vspace{0.15cm}

\textbf{Pre-condición}\par
El usuario ha iniciado sesión en su cuenta en Mini PIM y no tiene la máxima capacidad de relaciones creadas establecidas su plan de suscripción\footnote{Todos los casos de uso están pensados para seguir el plan gratuito. El total de relaciones permitidas en este plan son 3.}.\par
\vspace{0.15cm}

\textbf{Post-condición}
\begin{itemize}
    \item Caso de éxito: La relación es creada correctamente, se muestra en la lista de relaciones con el nombre ingresado.
    \item Caso mínimo: El sistema notifica al usuario el resultado de la acción de crear relación: exitosa o fallida.
\end{itemize}

\textbf{Prioridad: }
Alta
\vspace{0.15cm}

\textbf{Autores: }
Pablo Ortega Serapio y Francisco Javier Jordá Garay\par
\vspace{0.15cm}

\textbf{Control de cambios: } Versión 1: Definición del caso de uso

\numberedsubsection{Escenario principal}
\begin{enumerate}
    \item El usuario ha iniciado sesión con su cuenta de usuario correspondiente.
    \item El usuario accede a la sección de \enquote{Relaciones}.
    \item El sistema muestra la lista de relaciones existentes y la opción para crear una nueva.
    \item El usuario selecciona la opción \enquote{Crear relación} para crear una nueva relación.
    \item El sistema muestra el menú de creación de relación.
    \item El usuario ingresa el nombre deseado y selecciona los productos que quiere que se relacionen.
    \item El sistema verifica que el nombre de la relación no sea erróneo\footnote{Se considera erróneo cualquier nombre que este vacío, no sea único o sea igual a algún GTIN o algún SKU del sistema.}.
    \item El sistema verifica que si se selecciona un producto de la columna izquierda, también debe haber un producto seleccionado en la columna derecha para poder crear la relación\footnote{No permitimos en nuestro sistema relacionar un producto a \emph{null}. O la relación está vacía (sin productos seleccionas) o si se selecciona un producto, debe tener como mínimo alguno relacionado.}.
    \item El sistema verifica que cuando hay un elemento seleccionado en la columna izquierda el elemento seleccionado en la columna derecha no es el mismo.
    \item El sistema crea la nueva relación y actualiza la interfaz.
    \item El sistema muestra la lista de relaciones con la nueva relación añadida.
\end{enumerate}

\numberedsubsection{Escenarios alternativos}
\begin{description}

    \item[4.a] El sistema no permite crear relaciones.
    \begin{enumerate}
        \item[4.a.1] El sistema no permite crear relaciones adicionales inhabilitando el botón de \enquote{Crear relación} hasta que se eliminen relaciones existentes.
    \end{enumerate}
    
    \item[7.a] El sistema detecta que el nombre de la relación es erróneo.
    \begin{enumerate}
        \item[7.a.1] El sistema muestra un mensaje de error indicando que el nombre no es válido.
        \item[7.a.2] El sistema devuelve al usuario al menú de creación de relación permitiendo modificar el nombre.
    \end{enumerate}
    
    \item[8.a] El sistema detecta que solo hay elementos seleccionados en la columna izquierda.
    \begin{enumerate}
        \item[8.a.1] El sistema muestra un mensaje de error solicitando que seleccione un producto de la tabla derecha o que se deseleccionen los productos de la columna izquierda.
        \item[8.a.2] El sistema devuelve al usuario al menú de creación de relación para continuar eligiendo productos.
    \end{enumerate}

    \item[9.a] El sistema detecta que el producto seleccionado en la columna derecha es el mismo que el de la columna izquierda.
    \begin{enumerate}
        \item [9.a.1] El sistema muestra un mensaje de error indicando que no puede relacionarse un producto consigo mismo.
        \item [9.a.2] El sistema regresa a la sección de \enquote{Relaciones} sin generar ningún cambio.
    \end{enumerate}

    \item[*.a] El usuario cancela la acción de crear una relación nueva cerrando el menú de creación de relación.
    \begin{enumerate}
        \item[*.a.1] El sistema regresa a la sección de \enquote{Relaciones} sin generar ningún cambio.
    \end{enumerate}
\end{description}

\numberedsubsection{Casos de Prueba}
\underline{Escenario: Principal}\par
\vspace{0.15cm}
\textbf{Dado} que el usuario ha iniciado sesión con su cuenta de usuario correspondiente,\par
\textbf{Y} no ha alcanzado el número máximo de relaciones creadas,\par
\textbf{Y} está en el apartado de \enquote{Relaciones},\par
\textbf{Cuando} selecciona la opción \enquote{Crear relación},\par
\textbf{E} ingresa correctamente el nombre de la relación que desea crear,\par
\textbf{Y} selecciona correctamente los productos en las columnas de izquierda y derecha,\par
\textbf{Y} selecciona \enquote{Confirmar} para guardar los datos,\par
\textbf{Entonces} el sistema almacena la información de la nueva relación,\par
\textbf{Y} actualiza la lista de relaciones para incluir la nueva relación creada,\par
\textbf{Y} muestra el apartado de \enquote{Relaciones} con todas las relaciones.\par

\vspace{0.20cm}

\underline{Escenario: Alternativo 4.a (límite de relaciones alcanzado)}\par
\vspace{0.15cm}
\textbf{Dado} que el usuario ha iniciado sesión con su cuenta de usuario correspondiente,\par
\textbf{Y} ha alcanzado el número máximo de relaciones creadas,\par
\textbf{Y} está en el apartado de \enquote{Relaciones},\par
\textbf{Cuando} el usuario intenta crear una nueva relación,\par
\textbf{Entonces} el sistema le inhabilita el botón de \enquote{Crear relación},\par
\textbf{Y} no le permite crear más relaciones hasta que se eliminen relaciones existentes.\par

\newpage % Para que se vea el Escenario: Alternativo 7.a en una sola página

\underline{Escenario: Alternativo 7.a (nombre de relación erróneo)}\par
\vspace{0.15cm}
\textbf{Dado} que el usuario ha iniciado sesión con su cuenta de usuario correspondiente,\par
\textbf{Y} no ha alcanzado el número máximo de relaciones creadas,\par
\textbf{Y} está en el apartado de \enquote{Relaciones},\par
\textbf{Cuando} el usuario selecciona la opción \enquote{Crear relación},\par
\textbf{E} ingresa un nombre erróneo para la relación,\par
\textbf{Entonces} el sistema notifica al usuario que el nombre de la relación es erróneo,\par
\textbf{Y} muestra un mensaje de error indicando que el nombre no es válido,\par
\textbf{Y} devuelve al usuario al menú de creación de relación permitiendo modificar el nombre.\par

\vspace{0.20cm}

\underline{Escenario: Alternativo 8.a (productos no seleccionados correctamente)}\par
\vspace{0.15cm}
\textbf{Dado} que el usuario ha iniciado sesión con su cuenta de usuario correspondiente,\par
\textbf{Y} no ha alcanzado el número máximo de relaciones creadas,\par
\textbf{Y} está en el apartado de \enquote{Relaciones},\par
\textbf{Cuando} el usuario selecciona la opción \enquote{Crear relación},\par
\textbf{Y} selecciona productos de forma errónea,\par
\textbf{Entonces} el sistema muestra un mensaje de error indicando que debe seleccionar productos correctamente en ambas columnas,\par
\textbf{Y} devuelve al usuario al menú de creación de relación para continuar seleccionando productos.\par

\vspace{0.20cm}

\underline{Escenario: Alternativo 9.a (mismo producto seleccionado en ambas columnas)}\par
\vspace{0.15cm}

\textbf{Dado} que el usuario ha iniciado sesión con su cuenta de usuario correspondiente,\par
\textbf{Y} se encuentra en el apartado de \enquote{Relaciones},\par
\textbf{Cuando} selecciona la opción de \enquote{Crear relación},\par
\textbf{Y} selecciona el mismo producto en ambas columnas,\par
\textbf{Entonces} el sistema muestra un mensaje de error al usuario indicando que un producto no puede relacionarse consigo mismo,\par
\textbf{Y} muestra la lista de relaciones sin realizar ningún cambio.\par

\vspace{0.20cm}

\underline{Escenario: Alternativo *.a (cancelación de la creación de una nueva relación)}\par
\vspace{0.15cm}
\textbf{Dado} que el usuario ha iniciado sesión con su cuenta de usuario correspondiente,\par
\textbf{Y} no ha alcanzado el número máximo de relaciones creadas,\par
\textbf{Y} está en el apartado de \enquote{Relaciones},\par
\textbf{Cuando} el usuario selecciona la opción \enquote{Crear relación},\par
\textbf{Y} el usuario decide cancelar la creación de la relación seleccionando la opción de cerrar el menú de creación,\par
\textbf{Entonces} el sistema regresa al apartado de \enquote{Relaciones},\par
\textbf{Y} muestra el listado de relaciones existentes sin ningún cambio realizado.\par

\vspace{0.20cm}

\numberedsubsection{Bocetos}
\begin{figure}[H]
    \includegraphics[width=1\linewidth]{assets/mockups/RF5.1_1.png}
    \caption{Sección de \enquote{Relaciones}}
   \end{figure}
\vspace{1.0cm}

\begin{figure}[H]
    \includegraphics[width=1\linewidth]{assets/mockups/RF5.1_2.png}
    \caption{Menú de creación de relación tras clicar la opción \enquote{Crear relación}}
   \end{figure}
\vspace{1.0cm}

\begin{figure}[H]
    \includegraphics[width=1\linewidth]{assets/mockups/RF5.1_3.png}
    \caption{Escenario con al máximo de relaciones creadas}
   \end{figure}
\vspace{1.0cm}

\newpage % Muestra el diagrama de secuencias en una nueva página

\numberedsubsection{Diagrama de Secuencia}
\begin{figure}[H]
    \includegraphics[width=1\linewidth]{assets/sequence/Crear.jpg}
    \caption{Diagrama de Secuencia para crear una Relación}
   \end{figure}
\vspace{1.0cm}

\newpage % Inicia en una nueva página otro caso de uso