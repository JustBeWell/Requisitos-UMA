\phantomsection

\numberedsection{RF6.4 Borrar Atributo}

\subsection*{Descripción}
Los usuarios pueden borrar los atributos de usuario ya existentes en la cuenta.
\vspace{0.15cm}

\textbf{Pre-condición}\par
El usuario ha iniciado sesión en su cuenta en Mini PIM, tiene acceso a la sección de gestión de atributos y hay al menos un atributo de usuario creado.\par
\vspace{0.15cm}

\textbf{Post-condición}
\begin{itemize}
    \item Caso de éxito: El atributo es borrado y eliminado de la base de datos.
    \item Caso mínimo: El sistema notifica al usuario el resultado de la acción borrar atributo; exitosa o fallida.
\end{itemize}

\textbf{Prioridad: }
Alta
\vspace{0.15cm}

\textbf{Autor(es): }
Diego Sicre y Pablo Ortega\par
\vspace{0.15cm}

\textbf{Control de cambios: }
\begin{itemize}
    \item Versión 1: Definición del caso de uso.
\end{itemize}


\numberedsubsection{Escenario principal}
\begin{enumerate}
    \item El usuario selecciona el atributo que quiere borrar.
    \item El usuario pulsa el icono de borrar.
    \item El sistema pregunta al usuario si está seguro de su decisión.
    \item El usuario confirma seleccionando "Sí".
    \item El sistema borra el atributo de la base de datos.
    \item El sistema muestra al usuario un mensaje confirmando que el atributo ha sido borrado con éxito.
\end{enumerate}

\numberedsubsection{Escenarios alternativos}
\begin{description}

    \item[4.a] El usuario decide no borrar el atributo.
    \begin{enumerate}
        \item[4.a.1] El usuario selecciona "No" en el mensaje de confirmación.
        \item[4.a.2] El sistema cancela la acción de borrado y vuelve a mostrar la lista de atributos sin cambios.
    \end{enumerate}

\end{description}

\numberedsubsection{Casos de Prueba}
\underline{Escenario: Principal}\par
\vspace{0.15cm}
\textbf{Dado} que el usuario ha iniciado sesión en su cuenta en Mini PIM,\par
\textbf{Y} se encuentra en la sección de atributos,\par
\textbf{Y} selecciona un atributo,\par
\textbf{Cuando} pulsa el botón de borrar y confirma la acción,\par
\textbf{Entonces} el sistema borra el atributo de la base de datos y vuelve a mostrar la lista de atributos.\par
\vspace{0.20cm}

\underline{Escenario: Alternativo 3.a}\par
\vspace{0.15cm}
\textbf{Dado} que el usuario ha iniciado la acción de borrar un atributo,\par
\textbf{Y} el sistema muestra un mensaje de confirmación,\par
\textbf{Cuando} el usuario selecciona "No" en la confirmación,\par
\textbf{Entonces} el sistema cancela la acción y muestra la lista de atributos sin cambios.\par
\vspace{0.20cm}
\numberedsubsection{Bocetos}
\begin{figure}[H]
    \includegraphics[width=1\linewidth]{mockups/RF6.4BorrarAtributo.png}
    \caption{Borrar Atributo}
   \end{figure}
\vspace{1.0cm}
\numberedsubsection{Bocetos}
\begin{figure}[H]
    \includegraphics[width=1\linewidth]{mockups/RF6.4BorrarAtributoDespuesDeBorrar.png}
    \caption{Borrar Atributos Paso 2}
   \end{figure}
\vspace{1.0cm}




\newpage %Inicia en una nueva página otro caso de uso